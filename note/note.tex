%! TEX TS-program = xelatex

\documentclass[a4paper,10pt]{article}

% Подключение библиотек
\usepackage{geometry}
\usepackage{lipsum}
\usepackage{fancyhdr}
\usepackage{fontspec}
\usepackage{setspace}
\usepackage{ulem}
\usepackage{indentfirst}
\usepackage[english, russian]{babel}
\usepackage[hidelinks]{hyperref}
\usepackage{graphicx}
\usepackage{amsmath}
\usepackage{totcount}
\usepackage{calc}
\usepackage{tabularx}


% Установка пути картинок
\graphicspath{ {./images/} }

% Установка базового шрифта (Требуется XeLatex)
\setmainfont{Times New Roman}

% Подключение файлов. В них возможны подключения библиотек. Поэтому выше список используемых  библиотек не полон.
% Файл с константами документа
% Константы

% Тема
\newcommand{\topic}{Разработка машины времени}
% Фамилия автора документа
\newcommand{\authorsecondname}{Волков}
% Фамилия с инициалами
\newcommand{\authorwithinitials}{
	\authorsecondname~М.~В.}


% ТПЖА
\newcommand{\tpga}{TПЖА.09.03.01.514~ПЗ}
% Направление
\newcommand{\specialization}{09.03.01 - Информатика и вычислительная техника}
% Профиль
\newcommand{\profile}{Программное и аппаратное обеспечение вычислительной техники}

% Студент. Группа
\newcommand{\studentgroup}{студент гр.ИВТб-4301-04-00}

% Заведующая кафедрой
\newcommand{\headofdepartment}{Долженкова М. Л.}

% Руководитель
\newcommand{\supervisor}{Долженкова М. Л.}

% Звание руководителя
\newcommand{\supervisorrank}{к.т.н., доцент}

% Нормоконтролер
\newcommand{\norminspector}{Скворцов А. А.}

% Звание нормоконтролера
\newcommand{\norminspectorrank}{к.т.н., доцент}

% Количество плакатов
\newcommand{\numberofposters}{8}

% Рамки %

% Кафедра и группа
\newcommand{\departmentandgroupinframe}{Кафедра ЭВМ Группа ИВТ-41}

% Разработал
\newcommand{\authorinframe}{Бушков}

% Проверяющий
\newcommand{\inspectorinframe}{Долженкова}

% Нормоконтролер
\newcommand{\norminspectorinframe}{Скворцов}

% Утверждающий
\newcommand{\approverinframe}{Долженкова}

% Файл с рамками
% Определение рамок
\usepackage{fontspec}
\newcommand{\arial}{\fontspec{Arial}}
\usepackage{tikz}
\newcommand{\mainframe}[9]{
	\itshape
	\small
	\arial
	\begin{tikzpicture}[remember picture, overlay]

		\draw[black, ultra thick]

		([shift={(20mm, 5mm)}] current page.south west)
		--
		([shift={(20mm, -5mm)}] current page.north west)
		--
		([shift={(-5mm, -5mm)}] current page.north east)
		--
		([shift={(-5mm, 5mm)}] current page.south east)
		-- cycle;
		\draw[black, ultra thick]
		([shift={(20mm, 45mm)}] current page.south west)
		--
		([shift={(-5mm, 45mm)}] current page.south east)
		([shift={(20mm, 35mm)}] current page.south west)
		--
		++(65mm, 0)
		([shift={(20mm, 30mm)}] current page.south west)
		--
		([shift={(-5mm, 30mm)}] current page.south east)
		++(0, -5mm)
		--
		+(-50mm, 0)
		++(0, -5mm)
		--
		+(-50mm, 0)

		([shift={(20mm, 45mm)}] current page.south west)
		++(7mm, 0)
		--
		+(0mm, -15mm)
		++(10mm, 0)
		--
		+(0, -40mm)
		++(23mm, 0)
		--
		+(0, -40mm)
		++(15mm, 0)
		--
		+(0, -40mm)
		++(10mm, 0)
		--
		+(0, -40mm)

		([shift={(-5mm, 30mm)}] current page.south east)
		++(-20mm, 0)
		--
		+(0, -10mm)
		++(-15mm, 0)
		--
		+(0, -10mm)
		++(-15mm, 0)
		--
		+(0, -25mm);

		\draw[black, thick]
		([shift={(-5mm, 30mm)}] current page.south east)
		++(-40mm, -5mm)
		--
		+(0, -5mm)
		++(-5mm, 0)
		--
		+(0, -5mm);
		\draw[black, thick]
		([shift={(20mm, 5mm)}] current page.south west)
		++(0, 5mm)
		--
		+(65mm, 0)
		++(0, 5mm)
		--
		+(65mm, 0)
		++(0, 5mm)
		--
		+(65mm, 0)
		++(0, 5mm)
		--
		+(65mm, 0)
		++(0, 15mm)
		--
		+(65mm, 0);
		\draw[anchor=mid]
		([shift={(20mm, 35mm)}] current page.south west)
		++(0, -2.7mm)
		+(3.5mm, 0)
		node {Изм.}

		+(12mm, 0)
		node {Лист}

		+(28.5mm, 0)
		node {№ докум.}

		+(47.5mm, 0)
		node {Подп.}

		+(60mm, 0)
		node {Дата}

		+(8.5mm, -5mm)
		node [
				text width = 15mm,
				align = left
			] {Разраб.}

		+(8.5mm, -10mm)
		node [
				text width = 15mm,
				align = left
			]{Пров.}

		+(8.5mm, -15mm)
		node [
				text width = 15mm,
				align = left
			] {Реценз.}

		+(8.5mm, -20mm)
		node [
				text width = 15mm,
				align = left
			] {Н. контр.}

		+(8.5mm, -25mm)
		node [
				text width = 15mm,
				align = left
			] {Утв.}





		([shift={(-5mm, 30mm)}] current page.south east)
		++(0, -2.7mm)
		+(-10mm, 0)
		node {Листов}

		+(-27.5mm, 0)
		node {Лист}

		+(-42.5mm, 0)
		node {Литера}

		++(0, -5mm)
		+(-10mm, 0)
		node {#9}

		+(-27.5mm, 0)
		node {#8}
		;


		\draw[anchor=mid]
		([shift={(37mm, 30mm)}] current page.south west)

		+(11.5mm, -2.7mm)
		node [
				text width = 21mm,
				align = left
			] {#4}

		+(11.5mm, -7.7mm)
		node [
				text width = 21mm,
				align = left
			] {#5}

		+(11.5mm, -17.7mm)
		node [
				text width = 21mm,
				align = left
			] {#6}

		+(11.5mm, -22.7mm)
		node [
				text width = 21mm,
				align = left
			] {#7}
		;

		%\upshape
		\huge
		\draw
		([shift={(-5mm, 30mm)}] current page.south east)
		+(-60mm, 7.5mm)
		node {#1}
		;

		\large
		\draw
		([shift={(-5mm, 5mm)}] current page.south east)
		+(-25mm, 7.5mm)
		node [
				anchor = center,
				text width = 50mm,
				align = center
			]{#2}
		;

		\large
		\draw
		([shift={(-55mm, 5mm)}] current page.south east)
		+(-35mm, 12.5mm)
		node [
				anchor = center,
				text width = 70mm,
				align = center
			]{#3}
		;

	\end{tikzpicture}
}




\newcommand{\pageframe}[2]{
	\itshape
	\small
	\arial
	\begin{tikzpicture}[remember picture, overlay]

		\draw[black, ultra thick]

		([shift={(20mm, 5mm)}] current page.south west)
		--
		([shift={(20mm, -5mm)}] current page.north west)
		--
		([shift={(-5mm, -5mm)}] current page.north east)
		--
		([shift={(-5mm, 5mm)}] current page.south east)
		-- cycle;
		\draw[black, ultra thick]
		([shift={(20mm, 20mm)}] current page.south west)
		--
		([shift={(-5mm, 20mm)}] current page.south east)

		([shift={(20mm, 10mm)}] current page.south west)
		--
		+(65mm, 0)

		([shift={(-5mm, 5mm)}] current page.south east)
		+(0, 8mm) -- +(-10mm, 8mm)
		+(-10mm, 0) -- +(-10mm, 15mm)

		([shift={(20mm, 5mm)}] current page.south west)
		++(7mm, 0) -- +(0, 15mm)
		++(10mm, 0) -- +(0, 15mm)
		++(23mm, 0) -- +(0, 15mm)
		++(15mm, 0) -- +(0, 15mm)
		++(10mm, 0) -- +(0, 15mm)
		;

		\draw[black, thick]
		([shift={(20mm, 15mm)}] current page.south west)
		-- +(65mm, 0)
		;

		\draw[anchor=mid]
		([shift={(20mm, 5mm)}] current page.south west)
		++(0, 2.3mm)
		+(3.5mm, 0)
		node {Изм.}

		++(7mm, 0)
		+(5mm, 0)
		node {Лист}

		++(10mm, 0)
		+(11.5mm, 0)
		node {№ докум.}

		++(23mm, 0)
		+(7.5mm, 0)
		node {Подп.}

		++(15mm, 0)
		+(5mm, 0)
		node {Дата}

		([shift={(-5mm, 13mm)}] current page.south east)
		+(-5mm, 3.3mm)
		node {Лист}
		;

		\normalsize
		\draw
		([shift={(-10mm, 9mm)}] current page.south east)
		node {#2}
		;

		\huge
		\draw
		([shift={(85mm, 5mm)}] current page.south west)
		+(55mm, +7.5mm)
		node {#1}
		;

	\end{tikzpicture}
}

% Файл настроек / стилизации таблицы содержания ("Содержание")
% Изменение стилей таблицы содержимого
\usepackage{tocloft}

\setcounter{tocdepth}{5}

\renewcommand{\cftsecleader}{\cftdotfill{\cftdotsep}}
\renewcommand{\cftdotsep}{1}
\cftsetrmarg{0pt}
\renewcommand{\cfttoctitlefont}{}


\renewcommand{\cftsecpagefont}{}
\renewcommand{\cftsubsecpagefont}{}
\renewcommand{\cftsubsubsecpagefont}{}
\renewcommand{\cftparafont}{}

%\newcommand{\secnumwidth}{1cm}
%\newcommand{\subsecnumwidth}{1.5cm}
%\newcommand{\subsubsecnumwidth}{2cm}
%\newcommand{\paranumwidth}{2cm}


\setlength{\cftsecnumwidth}{2em}
\setlength{\cftsubsecnumwidth}{3em}
\setlength{\cftsubsubsecnumwidth}{4em}
\setlength{\cftparanumwidth}{5em}

\newcommand{\secindent}{0em}
\newcommand{\subsecindent}{1em}
\newcommand{\subsubsecindent}{2em}
\newcommand{\paraindent}{3em}

\setlength{\cftsecindent}{-\cftsecnumwidth}
\setlength{\cftsubsecindent}{-\cftsubsecnumwidth}
\setlength{\cftsubsubsecindent}{-\cftsubsubsecnumwidth}
\setlength{\cftparaindent}{-\cftparanumwidth}

\renewcommand{\cftsecfont}{\hspace{\cftsecnumwidth}\hspace{\secindent}}
\renewcommand{\cftsubsecfont}{\hspace{\cftsubsecnumwidth}\hspace{\subsecindent}}
\renewcommand{\cftsubsubsecfont}{\hspace{\cftsubsubsecnumwidth}\hspace{\subsubsecindent}}
\renewcommand{\cftparafont}{\hspace{\cftparanumwidth}\hspace{\paraindent}}


\setlength{\cftbeforesecskip}{0.5em}
\setlength{\cftbeforesubsecskip}{0.5em}
\setlength{\cftbeforesubsubsecskip}{0.5em}
\setlength{\cftbeforeparaskip}{0.5em}




% Файл стилизации названий разделов
% Изменение стилей названий разделов
\usepackage{titlesec}

\setcounter{secnumdepth}{4}

\titleformat{\section}[block]{\hspace{\parindent}}{\thesection}{1em}{}
\titleformat{\subsection}[block]{\hspace{\parindent}}{\thesubsection}{1em}{}
\titleformat{\subsubsection}[block]{\hspace{\parindent}}{\thesubsubsection}{1em}{}
\titleformat{\paragraph}[block]{\hspace{\parindent}}{\theparagraph}{1em}{}


\titlespacing{\section}{0pt}{2em}{2em}
\titlespacing{\subsection}{0pt}{2em}{2em}
\titlespacing{\subsubsection}{0pt}{2em}{2em}
\titlespacing{\paragraph}{0pt}{2em}{2em}

% Файл, в котором содержаться пользовательские команды
% Пользовательские команды

% Команда определения заголовков разделов, которые без нумерации и по середине страницы
\newcommand{\csection}[1]{
	{
			\titleformat{\section}[block]{\centering}{\thesection}{1em}{}
			\phantomsection
			\section*{#1}
			\addcontentsline{toc}{section}{#1}
		}
}

% Файл стилизации описаний рисунков и таблиц
% Изменение стилей описаний рисунков, таблиц...
\usepackage{caption}

\DeclareCaptionLabelSeparator{custom}{ -- }
\DeclareCaptionLabelFormat{flformat}{Рисунок #2}
\DeclareCaptionLabelFormat{tlformat}{Таблица #2}


\captionsetup[figure]{
	font=Large,
	labelsep=custom,
	labelformat=flformat,
	justification=centering,
	margin=1cm,
	aboveskip=0.5cm,
	belowskip=0.5cm,
}


\captionsetup[table]{
	format=plain,
	font=Large,
	labelsep=custom,
	labelformat=tlformat,
	singlelinecheck=false,
	margin={\docparindent, 0pt},
	skip=0.5em,
}

% Файл стилизации списков
% Изменение стилей списков

\usepackage{enumitem}



\newcommand{\labelv}{--}
% Отступ от label у itemize
\newcommand{\ilabelsep}{0.5em}
% Отступ от label y enumerate, level 1
\newcommand{\eilabelsep}{0.5em}
% Отступ от label у enumarate, level 2
\newcommand{\eiilabelsep}{0em}


\setlist[itemize]{
	itemsep=0pt,
	parsep=0pt,
	topsep=0pt,
	labelsep=\ilabelsep,
	label=\labelv,
	itemindent=\parindent+\ilabelsep+\labelwidth,
	leftmargin=0pt,
	align=left,
}


\setlist[enumerate,1]{
	label=\arabic*),
	ref=\arabic*,
	align=left,
	itemsep=0pt,
	parsep=0pt,
	topsep=0pt,
	labelwidth=\widthof{99)},
	labelsep=\eilabelsep,
	leftmargin=\parindent+\labelwidth+\labelsep,
}

\setlist[enumerate,2]{
	label=\theenumi.\arabic*),
	align=left,
	itemsep=0pt,
	parsep=0pt,
	topsep=0pt,
	labelsep=\eiilabelsep,
	labelwidth=\widthof{99.99)},
	leftmargin=\labelwidth+\labelsep,
}




%\renewcommand{\cftchapleader}{\cftdotfill{\cftdotsep}}

% Настройка отступов от краёв страницы до текста
\geometry{top = 15mm, left = 25mm, right = 10mm, bottom = 15mm}

% Убираем линию у верхнего колонтитула
\renewcommand{\headrulewidth}{0pt}

% Определяем стиль главной рамки
\fancypagestyle{mainframe} {
    \fancyhf{}
    \fancyhead[L]{
        \mainframe
        {\tpga}
        {Кафедра ЭВМ Группа ИВТ-41}{
        \topic
        }
        {\authori}
        {Долженкова}
        {Скворцов}
        {Долженкова}
        {\thepage}
        {\total{page}}
    }

}
% Определяем стиль рамки для страниц
\fancypagestyle{pageframe} {
    \fancyhf{}
    \fancyhead[L]{
        \pageframe
        {\tpga}
        {\thepage}
    }
}


%\setlength{\textfloatsep}{1cm plus 0.5cm minus 0.5cm}

% Межстрочный интервал, равный 1.5
\onehalfspacing

% Устанавливаем отступ абзаца (Значение берется из файла констант)
\setlength{\parindent}{\docparindent}

% Выравнивание по ширине 
% (Если не влезает текст, то увеличивает отступы между словами)
\sloppy

% Регистрация счётчиков для подсчёта разных объектов документа(страниц, формул, таблиц...)
\regtotcounter{page}
\regtotcounter{equation}
\regtotcounter{figure}

% Начало документа
\begin{document}
% Меняем название заголовка таблицы содержимого
\renewcommand{\contentsname}{\hfill Содержание \hfill}
% Устанавливаем размер шрифта 14pt

\Large
% Устанавливаем стиль пустой, чтобы убрать дефолтную нумерацию страниц
\pagestyle{empty}
% Подключаем титульник
% Титульник не изменяет счётчик страниц, он как бы не участвует в нумерации страниц,
% поэтому нумерация страниц начнётся со страницы "Реферат". 
% В итоге как бы реферат не будет участвовать в нумерации, что нам и нужно
\begin{titlepage}
	\newpage
	\singlespacing
	\topskip = 0.8cm
	\large
	%\setlength{\ULdepth}{1.8pt}

	\begin{center}
		\large
		%\fontsize{11pt}{13pt}
		\bfseries
		МИНИСТЕРСТВО НАУКИ И ВЫСШЕГО ОБРАЗОВАНИЯ РФ \\
		ФЕДЕРАЛЬНОЕ ГОСУДАРСТВЕННОЕ БЮДЖЕТНОЕ \\
		ОБРАЗОВАТЕЛЬНОЕ УЧРЕЖДЕНИЕ ВЫСШЕГО ОБРАЗОВАНИЯ \\
		«ВЯТСКИЙ ГОСУДАРСТВЕННЫЙ УНИВЕРСИТЕТ» \\
		ИНСТИТУТ МАТЕМАТИКИ И ИНФОРМАЦИОННЫХ СИСТЕМ \\
		ФАКУЛЬТЕТ АВТОМАТИКИ И ВЫЧИСЛИТЕЛЬНОЙ ТЕХНИКИ \\
		КАФЕДРА ЭЛЕКТРОННЫХ ВЫЧИСЛИТЕЛЬНЫХ МАШИН
	\end{center}

	\vspace{0.8cm}
	\begin{center}
		\textbf{Направление}
		\uline{\specialization}

		\small
		\textit{(код и наименование направления)}

		\large
		Профиль – \uline{\profile}
	\end{center}

	\vspace{0.8cm}
	\begin{flushright}
		{
			\onehalfspacing
			Допускаю к защите \\
			Заведующий кафедрой ЭВМ \\
		}
		\vspace{1mm}
		\uline{\hspace{3cm}} / \uline{\headofdepartment} / \\
		\vspace{1mm}
		\small
		\itshape
		(подпись) \hspace{1.8cm} (Ф.И.О.) \hspace{1.4cm}

	\end{flushright}

	\vspace{1.5cm}
	\begin{center}
		\huge
		\bfseries
		\topic
	\end{center}
	\vspace{0pt}
	\begin{center}
		Пояснительная записка выпускной квалификационной работы \\
		\tpga
	\end{center}

	\newcommand{\ulinesize}{2.5cm}

	\large
	\vspace{1cm}
	\noindent
	Разработал: \studentgroup \hfill \uline{\hspace{\ulinesize}}
	/ \uline{\authorwithinitials} / \hspace{8mm} \uline{\hspace{\ulinesize}}

	\vspace{1.5cm}
	\noindent
	Руководитель: \supervisorrank
	\hfill \uline{\hspace{\ulinesize}}
	/ \uline{\supervisor} / \uline{\hspace{\ulinesize}}

	\vspace{1.5cm}
	\noindent
	Нормоконтролер: \norminspectorrank
	\hfill \uline{\hspace{\ulinesize}}
	/ \uline{\norminspector} / \hspace{4mm} \uline{\hspace{\ulinesize}}

	{
		\small
		\itshape
		\hfill
		(подпись) \hspace{1.6cm} (Ф.И.О) \hspace{2.2cm} (дата) \hspace{0.8cm}
	}

	\begin{center}
		\vfill
		Киров \the\year
		\vspace{1cm}
	\end{center}


\end{titlepage}


% Меняем размер нижнего отступа до текста, чтобы текст не заезжал на главную рамку. 
% (Отступ текста до рамки 10mm)
% Меняем как бы на странице "Реферат", чтобы изменения были задействованы на странице 
% "Содержание". 
\addtolength{\textheight}{-40mm}

% Подключаем страницу "Реферат"

{

\topskip = 0.8cm
\begin{center}
	Реферат
\end{center}

\vspace{1em}

\authorwithinitials\
\topic
:\
\mbox{\tpga}\
ВКР / ВятГУ, каф. ЭВМ; рук.
\supervisor – Киров, \the\year. –
Гр.ч. \numberofposters л. ф.А1;
ПЗ
\total{page} с.,
\total{figure} рис.,
1 табл.,
\total{equation} форм.,
1 источников,
1 прил.

\vspace{1.5em}


МАШИНА ВРЕМЕНИ,
МАШИНА,
ВРЕМЯ,
УСТРОЙСТВО,
ПУТЕШЕСТИВЕ ВО ВРЕМЕНИ,
БУДУЩЕЕ, ПРОШЛОЕ,
C++,
C.


\vspace{1.5em}

Объект - Машина времени, устройство или технология, позволяющая перемещаться во времени, как в прошлое, так и в будущее.

Цель - Создание машины времени имеет множество возможных целей, включая исследование и изучение исторических событий, изменение хода истории, исправление ошибок или предотвращение негативных событий, получение информации из будущего для решения проблем настоящего, и т.д.

Результат - Результат создания машины времени может быть разным в зависимости от целей, методов и последствий использования. Возможны различные сценарии, такие как изменение прошлого с созданием параллельных временных линий, возможные временные парадоксы, потенциальные изменения будущего и т.д. Создание машины времени представляет сложную и фантастическую тему, вызывающую много вопросов и интерпретаций.


}


% Устанавливаем главную рамку для одной страницы
\tocloftpagestyle{mainframe}

% Устанавливаем рамку страниц, которая будет отрисовываться после главной
\pagestyle{pageframe}

% Меняем размер нижнего отступа до текста, чтобы текст до рамки страниц был 10mm.
% Рамка страниц имеет меньшую высоту содержимого нижней таблицы.
\addtolength{\textheight}{+25mm}
% Устанавливаем "Содержание", включающее в себя описание разделов документа
\tableofcontents\newpage

% Подключаем содержимое документа
Объект - Машина времени, устройство или технология, позволяющая перемещаться во времени, как в прошлое, так и в будущее.

Цель - Создание машины времени имеет множество возможных целей, включая исследование и изучение исторических событий, изменение хода истории, исправление ошибок или предотвращение негативных событий, получение информации из будущего для решения проблем настоящего, и т.д.

Результат - Результат создания машины времени может быть разным в зависимости от целей, методов и последствий использования. Возможны различные сценарии, такие как изменение прошлого с созданием параллельных временных линий, возможные временные парадоксы, потенциальные изменения будущего и т.д. Создание машины времени представляет сложную и фантастическую тему, вызывающую много вопросов и интерпретаций.



\end{document}
